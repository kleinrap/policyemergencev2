This chapter presents the approach that was used to code the forest fire world model. The model is based on the forest fire model presented in the Mesa project. It was modified slightly to add cells, add types of cells, add firefighters or prevention measures. This changes are limited as they only introduce probabilities of a fire happening or being suppressed. Furthermore, a timer was introduced to help the forest re-grow after a certain amount of time. This is used to be able to run the model infinitely.

%
\section{Calculation of the states}

The belief states are calculated for the truth agent. They are obtained using the following equations:

\begin{enumerate}
\item DC1 - Economy: A value of $1$ would mean that the map is filled with empty and camp site cells and there was no fire. A value of $-1$ would mean that the whole map is filled with burnt cells.
	\begin{equation}
	Economy = \frac{Tourism + Safety}{2}
	\end{equation}
\item DC2 - Environment: A value of $1$ would mean that the mao is covered in forest. A value of $-1$ would mean that the map is fully burnt.
	\begin{equation}
	Environment = \frac{Forest \text{ } size + Safety}{2}
	\end{equation}
\item PC1 - Forest size: A value of $1$ would mean the map is full of thick forest. A value of $-1$ would mean it is empty of all forests.
	\begin{equation}
	Forest \text{ } size = \frac{0.75 Thick + 0.25 Thin}{Total}
	\end{equation}
\item PC2 - Tourism: A value of $1$ would mean that the map is full of camps. A value of $-1$ would mean it is full of thick forest.
	\begin{equation}
	Tourism = \frac{0.75 Camp + 0.25 Thick}{Total}
	\end{equation}
\item PC3 - Safety: A value of $1$ would mean that there is no burnt land. A value of $-1$ would mean that everything has burnt.
	\begin{equation}
	Safety = \frac{Monitoring + Firefighters + Prevention - Camp - Thick}{5}
	\end{equation}
\item S1 - Camp sites: A value of $1$ would mean the map is covered with camps. A value of $-1$ would mean the map has no camps.
	\begin{equation}
	Camp = \frac{Camp}{Total}
	\end{equation}
\item S2 - Planting: A value of $1$ would mean that there area lot of thin forests. A value of $-1$ would mean there is a no thin forests.
	\begin{equation}
	Planting = \frac{Thin}{Total}
	\end{equation}
\item S3 - Monitoring: A value of $1$ would mean that the burning probability is of 10\% for thin forest and 100\% for thick forests. A value of $-1$ would mean the probability is of 0\% for both.
	\begin{equation}
	Monitoring = 0.1 - burning \text{ } probability
	\end{equation}
\item S4 - Firefighters: A value of $-1$ would mean that there is the maximum 50\% of firefighters extinguishing the fire. A value of $1$ would mean there is no change of extinguishing the fire.
	\begin{equation}
	Firefighters = 0.5 - firefighter \text{ } probability
	\end{equation}
\item S5 - Prevention: A value of 1 would mean the map is filled with empty cells. A value of $-1$ would mean the map has no empty cells.
	\begin{equation}
	Prevention = \frac{Empty}{Total}
	\end{equation}
\end{enumerate}

%
\section{The policy instruments}

A number of policy instruments are considered within the model. They are presented below. These instruments were obtained arbitrarily. The first ten instruments only affect one secondary issues in the belief hierarchy. For the last six, they affect a mix of secondary beliefs. Each instrument was chosen with its opposite. They are presented in \autoref{tab:policyInstruments}

\begin{table}
\begin{center}
\caption{Policy instruments used affecting the secondary beliefs of the agents. Impact 1 is regarding camp sites, impact 2 planting new forets, impact 3 monitoring, impact 4 firefighters and impact 5 fire prevention.}
\begin{tabular}{| c | c | c | c | c | c|} \hline
{\bfseries Instrument}
		& {\bfseries Impact 1}
				& {\bfseries Impact 2} 
						& {\bfseries Impact 3}
								& {\bfseries Impact 4}
										& {\bfseries Impact 5} 	\\ \hline
0 		& 0.5	& 0		& 0		& 0		& 0	\\ \hline
1 		& -0.5	& 0		& 0		& 0		& 0	\\ \hline
2		& 0		& 0.5	& 0		& 0		& 0	\\ \hline
3 		& 0		& -0.5	& 0		& 0		& 0	\\ \hline
4		& 0		& 0		& 0.5	& 0		& 0	\\ \hline
5 		& 0		& 0		& -0.5	& 0		& 0	\\ \hline
6 		& 0		& 0		& 0		& 0.5	& 0	\\ \hline
7		& 0		& 0		& 0		& -0.5	& 0	\\ \hline
8 		& 0		& 0		& 0		& 0		& 0.5\\ \hline
9 		& 0		& 0		& 0		& 0		& -0.5\\ \hline
10 		& 0		& 0.2	& 0.3	& 0		& 0.5\\ \hline
11 		& 0		& -0.2	& -0.3	& 0		& -0.5\\ \hline
12 		& -0.4	& 0.5	& 0.1		& -0.9	& -0.5\\ \hline
13 		& 0.4	& -0.5	& -0.1	& 0.9	& 0.5\\ \hline
14 		& -0.8	& 0		& 0		& 0.9	& 0	\\ \hline
15 		& 0.8	& 0		& 0		& -0.9	& 0	\\ \hline
\end{tabular}
\label{tab:policyInstruments}
\end{center}
\end{table}

The policies used for the three streams theories at the policy core levels are also shown in \autoref{tab:policies}.

\begin{table}
\begin{center}
\caption{Policy instruments affecting the policy core beliefs (only used for the three streams model). Impact 1 is regarding forest sizes, impact 2 tourism and impact 3 safety.}
\begin{tabular}{| c | c | c | c |} \hline
{\bfseries Instrument}
		& {\bfseries Impact 1}
				& {\bfseries Impact 2} 
						& {\bfseries Impact 3}
								\\ \hline
0 		& 0.5	& 0		& 0		\\ \hline
1 		& -0.5	& 0		& 0		\\ \hline
2		& 0		& 0.5	& 0		\\ \hline
3 		& 0		& -0.5	& 0		\\ \hline
4		& 0		& 0		& 0.5	\\ \hline
5 		& 0		& 0		& -0.5	\\ \hline


\end{tabular}
\label{tab:policies}
\end{center}
\end{table}
