This chapter presents the entire revised conceptualisation. It was revised after reflection on the actions. The formalisation is similarly revised in the next appendix.

\section{Common core}
\label{sec:conceptualisationCommonCoreRevised}

The policy emergence process is a process composed of several {\bfseries iterative stages}. Different iterative stages can be differentiated. There are iterative stages where the {\bfseries actors} present in the {\bfseries policy arena} interact with each other. The policy arena is the environment in which these connected actors evolve. Additionally, different types of actors are present within the policy arena. They are the {\bfseries policy makers}, the {\bfseries policy entrepreneurs} and the {\bfseries external parties}. The policy makers are the actors that have decision making power. This means that they can decide what is placed on the {\bfseries agenda} or what {\bfseries policy instrument} should be implemented. The former is performed in the round called the {\bfseries agenda setting} while the latter is perform during the round called the {\bfseries policy formulation}. They can also work to influence the {\bfseries beliefs} of the other actors present in the policy arena which are connected to them through their {\bfseries policy network}. The policy network is the network that connects the actors present in one policy arena. The actors are connected based on their awareness of one another. The likelihood of actors to interact with one another is dependent on their awareness of one another. Over time the policy network of the actors will deteriorate. To maintain their network, the actors are able to perform simple actions using their {\bfseries resources}. These resources represent their financial and political resources.

Each policy maker possesses a belief hierarchy that defines its beliefs on the issues being discussed in the policy arena. This hierarchy is composed of field wide beliefs at the top and beliefs on increasingly detailed issues when progressing downwards in the hierarchy. Example of issues are public transportation or the amount of plastic present with a size less than 2 millimetres present in the oceans. For each of these issues, the actors have an understanding of the current {\bfseries states of the world} and a goal for where they would like to see this issue be. Furthermore, each level in the hierarchy is connected to the next through causal relations up to the highest level. These causal relations help the actor understand how different issues affect each other and therefore how the world works.

The impact of the influence of the policy makers on the other actors is dependent on the resources spent by the policy makers, the difference in beliefs they have with the actor they are influencing and the affiliation of both influencing and influenced actor. However, before influencing other actors, a decision must be made on what influence should be performed and on which actor. The likelihood that a policy maker will influence another actor will vary depending on their {\bfseries political affiliation} in relation to the affiliation of the actor influenced, their perceived {\bfseries conflict level} on the issues concerned with the actor influenced, their policy network (awareness level) and whether the actor being influenced is a policy maker or another actor. Within the policy arena, actors will prefer influencing policy makers as they have decision making power (they decide on the agenda and the policy instrument to be implemented). The political affiliation of the actors relates them to specific {\bfseries constituencies}. Furthermore, it affects the likelihood of actors of different affiliation of interacting with one another. The resources available to the policy makers are dependent on the size of their affiliated constituencies. There is a conflict level for each issue represented in the beliefs of a policy maker. This conflict level is the difference in the views of the world of the policy maker and what the policy maker thinks the beliefs of the actors influenced are. This perception of actor B’s beliefs by actor A is referred to as the {\bfseries partial knowledge} of the actor A. The amount of conflict level will affect the likelihood of the two actors interacting with one another about certain issues. Actors are more likely to discuss issues for which they have a mild conflict level and less likely to discuss issues for which they have a high conflict level.

There are different types of actions which the policy makers can perform on the other actors. They can perform {\bfseries framing actions}. These actions are attempts to influence other actors on their beliefs of how the world works (the causal relations in their beliefs hierarchies). The policy makers can also {\bfseries directly influence} the actors on their beliefs for the issues. They can influence them on the way they see the world or on their aims of what the world should be. Because the {\bfseries attention span} of the actors in the policy arena is limited, these actors tend to only influence other actors on one selected issue. This issue is selected based on the appearance of {\bfseries urgency}. This urgency is determined by the actors based on the gap between the current states of the world and their aim but also in relation to the impact that this issue ultimately has on their normative beliefs. The ultimate goals of the policy makers being to reach their normative belief goals. 

The partial knowledge of the actors is updated whenever an influencing action is performed. Both the actor influencing and the actor being influenced have been part of the interaction and therefore get a better, but not perfect, understanding of each other’s beliefs on the topic of the interaction.

As mentioned earlier, the policy makers can also decide what is on the agenda. This is done by selecting the issue that they consider to be the most urgent to them. The issue that is ultimately placed on the agenda is the issue that most policy makers believe is urgent. The choice of issue on the agenda will go on to define which issues can be discussed in iterative stages looking at lower level issues in the belief hierarchies. The issues all have to be related to the issue on the agenda. This is however subjective to the actors depending on their personal beliefs. The process of agenda setting therefore helps narrow down the problems observed by the policy makers and to ultimately decide on a {\bfseries policy instrument} that can be used to address such problem.

The policy entrepreneurs are actors that seek to influence the beliefs of the policy makers primarily but also to influence the beliefs of the other actors within the policy arena through their policy network. Their aim is to push their beliefs onto the other actors present in the arena. Similarly to the policy makers, the policy entrepreneurs have a belief hierarchy, a political affiliation and use resources to influence other actors. These resources are also defined based on their constituencies. They can also perform different actions on the policy makers, external parties and other policy entrepreneurs. They can frame how the world works, influence on the beliefs of other actor’s view of the world and their goals. The likelihood of performing an action and its actual impact are determined in a similar way as for the policy makers.

The external parties are actors that have multiple roles within the policy arena. The external parties represent actors such as the media, but also research institute or academic institutions. They help inform other external parties, the policy makers, the policy entrepreneurs about the current state of the world. They can also influence other external parties, policy makers and policy entrepreneurs on their beliefs of how the world works, on issue states and on issue aims. Finally, they can influence the goals of the constituencies. Similarly to the policy makers, the external parties have a certain political affiliation and resources stemming from that affiliation.

The external parties have direct access to the states of the world. They can therefore transmit this information to the actors within the policy arena along with the constituencies. However, not all external parties are interested in all states of the world, some external parties are only interested in specific states. Furthermore, this transmission of the states is dependent on the policy network, and the affiliation of the external parties and the actors to whom they are transmitting this information. Different actors will perceive the external parties as more or less trust worthy depending on their affiliation.

The external parties can also influence other external parties, the policy makers and the policy entrepreneurs. They can do this through framing actions, influence on aim actions and influence on state actions that affect all actors present in the policy arena called blanket actions. The likelihood of performing a blanket action is defined the same way as for the policy makers. The impact is  obtained as shown earlier on. The impact is however spread on all actors concerned. 

Finally, the external parties can influence the goals of the different constituencies through blanket influence. This is similar to the blanket framing that is used on the actors within the policy arena but this time they affect the others' goals. The likelihood of an action being performed will depend on the difference in beliefs between the constituencies and the external party concerned and the affiliation of the external party performing this action. The impact will depend on the resources spent and the difference in beliefs.

Each constituency represent a sector of the electorate and is associated to a political affiliation. Each constituency represents a certain percentage of the total voter population. This representativeness of a constituency is what affects the amount of resources that is distributed to the different actors within the policy arena as mentioned previously. The constituencies can influence the policy makers present in the policy arena. The policy maker’s goal beliefs tend to move towards the goals of their constituencies to increase their chance of remaining in office.

A second round can also be identified as a policy formulation round. In this round, the aim is to adopt a policy instrument based on the chosen agenda such that it is implemented in the world. Policy instruments are the specific mechanisms policy makers put in place in order to impact the issues discussed within the iterative stages. During the policy formulation round, the actors influence each other’s beliefs similarly to the agenda setting round. The aim here is however different. Each actor selects a policy instrument based on its expected impact in the world. This impact is assessed by the actors based on their beliefs of the change the instrument will have on the states of each issues compared to their own goals. Because the actors have a limited attention span, they can only select what they perceive as being the most appropriate policy instrument. They decide that based on the impact that the instrument have on their beliefs and based on which issues need to be addressed the most urgently. The interactions of the different actors are the same as for the agenda setting round. The likelihood of an action happening is however now determined based on the expected impact that each instrument has on the urgent issues in the actors’ beliefs within the narrower field of issues selected in the agenda setting stage.

One main difference is that the policy formulation round does not end with the selection of an issue but the selection of a policy instrument. Furthermore, the policy instrument selected is only implemented if the number of policy makers that support that instrument is superior to the threshold required for the implementation of the instrument. This threshold can be a majority, a two third majority or unanimity.

The last round considered is a round that does not involve the actors. It is the world. In this round, the instrument that has been selected is implemented. The world is affected by that instrument and the states of the world change accordingly. This then has an impact on the belief of the actors present in the policy arena as the change in the states is conveyed by the external parties to the different actors.

Finally external events can be introduced in the model. Such external events can affect anything from the goal of specific agents for a specific issue to a new distribution of the constituencies. These external events are devised based on case studies. The modeller must then adopt the impact of an election into the model’s framework to appropriately represent that election.

\section{Three streams theory}
\label{sec:conceptualisation3SRevised}

The introduction of the three streams theory changes some of the aspects presented in the common core. These changes are performed to account for the concept of streams and of policy window. The iterative stages considered are the same as the iterative stages mentioned in the common core. The difference is present in what the actors present in the policy arena choose as an issue. The actors can now choose from a policy or a problem as called for in the three streams theory. The problems are obtained in the belief hierarchy of the actors while the policy are obtained in a {\bfseries policy hierarchy}. 

The policy hierarchy is a hierarchy containing policy instruments. These policy instruments have specific impacts on the issues present in the belief hierarchy of the actors. The concept of policy instrument is therefore the same as the one presented in the common core. Within the three streams theory, policy instruments are present in all iterative stages. Each of these impact is now subjective which means that they also represent beliefs on the effectiveness of the different policies. These beliefs can also be influenced by the different actors present in the policy arena.

Each actor first selects a policy or a problem. The problems are graded similarly to what was previously presented in the common core: based on the perceived urgency of the chosen issue. The policy are graded based on their impact on their associated issues. Whichever grade is the highest will be chosen by the actor. If the actor chooses a problem first, then s/he will have to choose an associated policy based on the effectiveness of the policies. Furthermore, all actions performed by that actor from now on will be on the beliefs of other actors concerning the selected problem. These actions re the same as the ones in the common core. If an actor chooses a policy first, s/he will have to choose an associated problem based on urgency. Similarly to the problem, all actions performed by that actor from now on will be on the beliefs of other actors concerning the selected policy.

The policy actions are similar to the problem actions presented in the common core for the different actors but with one change to the framing action. The aim of the actor is now to affect the impact beliefs of other actors in their policy network. The actions on policies are therefore akin to the framing actions related to the problems. This can only be done by actors having first selected a policy as mentioned earlier. There are some important caveats to this approach. First, the constituencies do not possess any beliefs on the policy hierarchy. This means that external parties cannot influence the constituencies if they have selected a policy first, they will then move on to their problem (which they selected based on the policy first chosen) and influence the constituencies on that problem. Second, the constituencies cannot influence the instrument hierarchy of the policy makers, the influence remains focused only on the belief hierarchy.

Additionally to the introduction of streams, the policy entrepreneurship model which is tied to the three streams theory introduces the concept of {\bfseries teams}. Teams are small, short-term groups of actors which try to influence other actors on a specific problem or policy. These teams are constituted by actors when they consider that a policy or a problem is very urgent and when they can find other actors that share their beliefs on that policy or problem and the urgency of it. Once a problem or policy has been sufficiently pushed (enough actors have been influenced), then the team will disband and the actor will return to acting separately. Teams use the same actions as the individual actors. They provide actors with a larger policy network. This is because the team network consists of the sum of the networks of all actors present in the team. The teams obtain their resources from their members based on their feeling of belonging within a team. The actions performed by the team have to be agreed by the entire team as team are non-centralised entities. This means that no actor is powerful enough to impose his/her will on the other actors. Teams can also perform influencing actions on their own members. This helps increase the cohesiveness of the team so that they can last longer within the policy arena. The actions performed by team members on team members are blanket framing, influence on goals an influence on states.

Although there are also a set of iterative stages in the three streams theory, the actions performed in both iterative stages are not different anymore. Because each actor must have chosen either a policy or a problem, the actions are exactly the same in both iterative stages for all actors present in the policy network. The difference remains in the fact that for the agenda setting iterative stages, an agenda is set at the end which defines what is discussed thereafter. The agenda is composed of a problem and a policy. Each is chosen as being the most selected by the policy makers. The problems and policies discussed thereafter must relate to the problem and policy on the agenda. They must therefore be on a lower level in their respective hierarchies. For the policy formulation, the policy instrument chosen by the policy maker is the one that meets the threshold requirement once again. This disregard the fact that a policy maker might have chosen a problem first.

\section{The advocacy framework coalition}
\label{sec:conceptualisationACFRevised}

The introduction of the advocacy coalition framework adds some different changes to the common core. The main addition is the concept of {\bfseries coalitions}. The coalitions are groups to which actors are assigned based on their normative beliefs. They are long term entities that perform actions on other actors or other coalitions to influence them into similar beliefs. Actors only change coalitions when their normative beliefs vary which happens only rarely. The actions performed by coalitions are similar to the ones performed by teams. However, coalitions are centralised entities. This means that the {\bfseries coalition} leader is the only actor that decides which actions are implemented. The coalition leader is the actor within a coalition that has the largest policy network. Finally, the number of coalitions present within a policy arena is usually limited to only a few according to the literature. 

The ACF also introduces the policy broker. Policy brokers are actors that look in their network to put in contact actors which are badly connected within the policy arena. Policy makers, policy entrepreneurs and external parties can all be elevated to the role of policy broker which grants them additional actions. Two actions are added. With the first action, the policy broker can connect two actors which are not connected (but to which s/he is connected). In the second action, the policy broker can raise the awareness level between two actors (if his/her own awareness level is high enough). To perform these actions, the policy brokers is provided with additional resources fitting with the strength of his/her policy network.

Finally, which action is chosen by the policy broker is decided based on the type of policy broker. Neutral policy brokers will select the actions that will lead to the largest impact. This means that connecting actors that are not yet connected will come first, followed by raising the awareness between actors that are already aware of each other. If the policy broker plays an advocacy role, then s/he will select actions with actors that share his beliefs.

\section{Feedback theory}
\label{sec:conceptualisationFeedbackRevised}

For the feedback theory, the policy instruments implemented by the actors are supplemented with feedback effects. The feedback effects have an additional specified effect on different parts of the world or of the policy arena. When choosing for a policy instruments, the actors are not aware of the feedback effects. There are three feedback effects considered.

Some feedback effects have an impact on the constitution of the constituencies. The percentage distribution of each constituency will be affected. Some feedback effects have an impact on the belief hierarchy of the actors opening new issues to consideration in the hierarchy or removing others. Finally, a third feedback effect relates to the link between constituencies and the resources of the actors within the model. This links is affected by this type of feedback effects.

\section{Diffusion theory}
\label{sec:conceptualisationDiffusionRevised}

The introduction of the diffusion theory requires the consideration of multiple policy subsystems. These subsystems can represent different nations or cities for example but they are all regarding the same issues. The diffusion theory specifies the impact that the actors in a subsystem might have on actors of another subsystem. The subsystems share specific relationships ranging from {\bfseries friendly} to {\bfseries competitive} while considering {\bfseries coercive} and {\bfseries dominant}. These relationships help define the actions that can be considered between the actors of the different subsystems. Each subsystem also possesses a certain {\bfseries status}. This status helps define the amount of resources that is provided to each of their actors.

The actors in the different subsystems are all part of a {\bfseries super-policy network}. This network connects actors between the different subsystems. Through this network actors can influence each other depending on the relationships between the arenas. An actor will consider all possible actions. The actions that is most likely to be performed is implemented.

The subsystems themselves are part of a network. Each subsystem has a link with another. Each of these links are one directional and can be of one type from friendly to competitive. Links that are coercive or dominant also gain a {\bfseries strength} attribute. This strength attribute defines the influence that a subsystem will have on another when actors from one arena influence actors in the other. The actions that actors can perform is defined by the link that links their respective subsystems.

In the case of friendly relationship, the actor will be able to perform all actions that are available within the common core. These are actions of framing, influence on states and goals for the beliefs of the other actor. The likelihood of performing these actions along with their impact is assessed the same way as presented previously. When considering the diffusion theory with the common core or the ACF, then the actions are only performed on the issues. When it is considered with the three streams theory, then problems and policies actions are possible.

In the case of a coercive or dominant relationship, the actors will force their own goal beliefs onto selected actors in the other subsystem. The likelihood of this forced influence to be selected will depend on the conflict level between the actors and the type of actors that is being influenced (policy makers are more likely to be influence). Considering actions with an equal impact, actions on a dominant relationship will be most likely to happen, followed by coercive actions and finally friendly actions. The actual impact will only include the amount of resources spent, the difference in beliefs between the two actors and the strength of the considered subsystems. Once again, for similar actions, the impact will be largest in a dominant relationship.

In the case of a competitive relationship, the actors within one subsystem will seek to get a better world that the ones in other subsystems. To achieve this, they will change their goals to match the ones of other subsystems. This is also done through interactions. The likelihood of performing such an action along with their impact will is calculated similarly to the actions between friendly subsystems. The main difference with respect to the impact is that the impact of the actions are on the actor performing the action (the actor influences him/herself based on the beliefs of an actor in another subsystem).