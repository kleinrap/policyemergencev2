This appendix is a summary of all of the concepts that are mentioned within the theories. It includes their corresponding concepts within the model created in this thesis. When there is no relation, then the concept has not yet been addressed and is therefore not mentioned in the table. Note that a third column is added to signify the policy making theories concepts that have been addressed in the conceptualisation and formalisation but are not yet present within the code.

The concepts with an asterisk are detailed further after the table.

\bgroup
\def\arraystretch{1}
\begin{longtable}{|p{0.5\linewidth} | p{0.1\linewidth} | p{0.1\linewidth}|}
\hline
Policy making theories concept		& Model			& Concep.		\\ \hline \hline
\multicolumn{3}{|c|}{The three streams theory} \\ \hline
Fluid participation				& \cmark			& \cmark				\\ \hline
Problem preferences				& \cmark			& \cmark				\\ \hline
Unclear technology				& \xmark			& \xmark				\\ \hline
Policy stream					& \cmark			& \cmark				\\ \hline
Value acceptability$^*$			& \cmark			& \cmark				\\ \hline
Technical feasibility$^*$			& \cmark			& \cmark				\\ \hline
Integration of the instrument		& \xmark			& \xmark				\\ \hline
Problem stream				& \cmark			& \cmark				\\ \hline
Indicators	$^*$					& \cmark			& \cmark				\\ \hline
Focusing event					& \xmark			& \xmark				\\ \hline
Feedback$^*$					& \cmark			& \cmark				\\ \hline
Load							& \cmark			& \cmark				\\ \hline
Politics stream					& \cmark			& \cmark				\\ \hline
Policy makers					& \cmark			& \cmark				\\ \hline
Policy entrepreneurs				& \cmark			& \cmark				\\ \hline
Policy entrepreneurs	time constraints& \cmark			& \cmark				\\ \hline
Policy window					& \cmark			& \cmark				\\ \hline
Team creation criteria			& \cmark			& \cmark				\\ \hline
Leading by example				& \xmark			& \xmark				\\ \hline
Independent streams			& \cmark			& \cmark				\\ \hline

\hline
\multicolumn{3}{|c|}{The advocacy framework coalition} \\ \hline
Subsystem					& \cmark 			& \cmark 				\\ \hline
Coalitions						& \cmark 			& \cmark 				\\ \hline
Coalitons influence policy makers	& \cmark			& \cmark				\\ \hline
Devil shift						& \xmark			& \xmark				\\ \hline
Limited amount of information		& \cmark			& \cmark				\\ \hline
Deep core beliefs$^*$	 		& \cmark			& \cmark				\\ \hline
Policy core beliefs				& \cmark			& \cmark				\\ \hline
Secondary beliefs				& \cmark			& \cmark				\\ \hline
Stable coalitions				& \cmark			& \cmark				\\ \hline
Actors show substantial consensus	& \cmark			& \cmark				\\ \hline
Secondary before policy core$^*$	& \xmark			& \xmark				\\ \hline
Bounded rationality$^*$			& \cmark 			& \cmark 				\\ \hline
Belief system					& \cmark 			& \cmark 				\\ \hline
Coalitions creation criteria			& \cmark 			& \cmark 				\\ \hline
External event					& \cmark 			& \cmark 				\\ \hline
Internal subsystem event			& \xmark 			& \xmark 				\\ \hline
Negotiated agreement			& \xmark 			& \xmark 				\\ \hline
Policy-oriented learning			& \cmark 			& \cmark 				\\ \hline
Policy learning is more likely with moderate level of conflicts		
							& \cmark 			& \cmark				\\ \hline
Learning is more likely in a prestigious forum$^*$
							& \cmark 			& \cmark 				\\ \hline
Quantitative problems are more conducive to policy learning$^*$
							& \xmark			& \xmark				\\ \hline
Problems involving natural systems are more conducive to policy learning
							& \xmark			& \xmark				\\ \hline
Accumulation of technical information does not change the view of opposing coalitions
							& \xmark			& \xmark				\\ \hline
Administrative agencies advocate for more moderate measures	$^*$		
							& \cmark 			& \cmark 				\\ \hline
Actors within purposive groups are more constrained in their expression beliefs and policy positions than actors from material groups
							& \xmark			& \xmark				\\ \hline
\hline
\multicolumn{3}{|c|}{The diffusion theory} \\ \hline
Leaning mechanism				& \cmark 			& \cmark 				\\ \hline
Imitation						& \xmark 			& \xmark 				\\ \hline
Normative pressure				& \cmark 			& \cmark 				\\ \hline
Competition					& \cmark 			& \cmark 				\\ \hline
Coercion						& \cmark 			& \cmark 				\\ \hline
\hline
\multicolumn{3}{|c|}{The feedback theory} \\ \hline
Meaning of citizenship			& \cmark 			& \cmark 				\\ \hline
Form of governance				& \xmark 			& \xmark 				\\ \hline
Power of groups				& \cmark 			& \cmark 				\\ \hline
Definition of policy problems		& \cmark 			& \cmark 				\\ \hline
The resource effect				& \cmark 			& \cmark 				\\ \hline
The interpretive effect			& \cmark 			& \cmark 				\\ \hline
\multicolumn{3}{|c|}{The policy entrepreneurhsip model} \\ \hline
Social acuity					& \cmark 			& \cmark				\\ \hline
Definition of problems			& \cmark 			& \cmark 				\\ \hline
Policy entrepreneurs should be ready to build teams
							& \cmark 			& \cmark				\\ \hline
Definition of policy problems		& \xmark			& \xmark				\\ \hline
\end{longtable}
\egroup 


Notes concerning the concepts in the advocacy framework coalition:

\begin{itemize}
\item The bounded rationality concepts: The agents are not introduced within the model with bounded rationality for say. However, because the agents are included within a large set of agents and each of these agents perfom actions, the overall resulting model can be considered to have agents with boudned rationality. 
\item Value acceptability, technical feasibility, indicators: these two concepts are not part of the model for say, they can however be incorporated by the modeller in the assessment of the impact of the policy instruments as inputs to the model. In future work, they could be dynamically introduced in the model for the policy instruments. In this way, all policy instruments could be influenced by the agents present in the policy arena.
\item Focusing event: Focusing events are not part of the model. However, they can be introduced by the modeller through the use of external events. Depending on the design of the external event, it can act as a focusing event leading to changes in the policy emergence process.
\item Feedback: Feedback in the sense of the feedback theory has not been implemented (it was conceptualised and formalised). Feedback is however present through the world simulation in the model. 
\item Deep core beliefs: They are considered as principle beliefs. Additional layers in the belief hierarchy would be needed to see the use of deep core beliefs which is not excluded by the conceptualisation presented here.
\item Secondary before policy core: This assumption is not taken into account in the implementation. However, and this was mentioned within the report, it can be introduced easily.
\item Learning is more likely in a prestigious forum: This is not present directly in the model but is introduced through external events which provide a boost in awareness to specific agents participating in a forum.
\item Administrative agencies advocate for more moderate measures: This is not excluded, it is dependent on the inputs from the modeller.
\end{itemize}