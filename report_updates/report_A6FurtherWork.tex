This appendix presents the list of items that has been thought of for further work.

This first list describes the actions that would be required to have a more complete model.

\begin{enumerate}
\item Use the appropriate equation for the creation of new policy network links (currently the 0.5 coefficient is missing).
\item Fix of the formalisation for the awareness decay in the case of team and coalition actions.
\item The implementation of the new formalisation
\item The formalisation of the policy broker concept and its implementation.
\item The implementation of the diffusion theory.
\item The implementation of the feedback theory.
\item The implementation of the external party issue selectiveness mentioned in the conceptualisation.
\item The testing of different partial belief hierarchies initialisation methods.
\item The introduction of full knowledge at the principle belief level.
\item The adaption of the code to allow for more or less than three affiliations.
\item The addition of the infrastructure to be able to save what actions are performed by who, with what impact and whom.
\item Provide an analysis of the policy hierarchy results from the three streams model.
\item Introduce a case study and attempt to find a consistent way of designing the connection between the world and policy emergence model along with appropriate initialisation of the policy network.
\item Perform a more complete and consistent experimentation set along with a broader analysis of the results.
\item Introduce a difference between technical and non-technical issues. According to the literature \citep{nohrstedt2010logic}, this can be important. Actors are more likely to agree on technical issues and disagree on non-technical issues.
\end{enumerate}

This second list describes some extensions or some further work that could be needed to extend the model:

\begin{enumerate}
\item Introduce the possibility to have instruments that have an impact over time. This would require the addition of the possibility to grade these instruments against one time impact instruments.
\item The introduction of a policy package tool. This could be an extension where the agents can create their own policy instruments or an extension that uses current models that build policy instruments (there would then only be a need to connect both models).
\item To enrich the model, it could be possible to introduced the three types of subsystem behaviour mentioned in the literature \citep{weible2008expert, nohrstedt2010logic}. These are the unitary subsystem, the collaborative subsystem and the adversarial subsystem. The introduction of such differences could affect the behaviour algorithms of the different actors or change specific weights of specific actions within the actors' algorithms.
\item Construct an in-browser live visualisation.
\item Provide a in-browser GUI for the initialisation of the model.
\end{enumerate}