After the concepts and the actor actions and interactions have been presented, it is possible to look at the different models themselves. These models use all of the elements presented in the previous chapter and combine them into model that can represent different views of the policy emergence process. The different models considered are therefore presented and their respective differences outlined in \autoref{sec:differentModels}. This is followed by a detailed explanation of each of the model starting with the backbone in \autoref{sec:backbone}, the bacbkone+ in \autoref{sec:backbone+}, the three streams theory in \autoref{sec:3S} and the advocacy coalition framework in \autoref{sec:ACF}. This chapter is then completed with an explanation of the two extensions that can be added to these four models: the feedback theory in \autoref{sec:feedbackExtension} and the diffusion theory in \autoref{sec:diffusionExtension}.

%Now that the different concepts have been introduced, it is possible to look at how these concepts are used together within the model. The aim of this chapter is to describe how the model will operate for the different policy making theories. For this, first the actions that each of the actors can perform are shown in \autoref{sec:actorBeliefActions} and \autoref{sec:actorNetworkingActions}. Then an explanation is provided on the steps that are taken for the backbone in \autoref{sec:backboneActions}, the steps used for the backbone+ in \autoref{sec:backbone+Actions}, the three streams theory in \autoref{sec:3SActions} and the advocacy coalition framework in \autoref{sec:ACFActions}. The extensions related to the feedback theory are presented in \autoref{sec:feedbackActions} while the ones used for the diffusion theory are presented in \autoref{sec:diffusionActions}.

%
\section{The Different Models}
\label{sec:differentModels}

To represent the policy emergence process and considering that all theories are not fully compatible, different models are constructed. In total four models are built from the conceptualisation that has been presented up to now. For these models, the feedback and diffusion theories are considered to be extensions to the other theories. The three streams theory and the ACF are considered to be full fledged theories that can be modelled as stand alone models. Note that, as mentioned in the validation of the approach, the three streams theory and the ACF are never modelled simultaneously. They are always considered separately.

The first model considered is called the backbone. This model represents the skeleton that is used to model all other theories. In that spirit, it contains several of the compatible concepts while having the least amount of concepts. The second model is called the bacbkone+. While the backbone contains the smallest amount of concepts, this model contains all concepts that are compatible between the different theories. This model serves as a stepping stone between the backbone and the other models. The third model is the three streams model. This model is an extension of the bacbkone+ model with the addition of the streams and teams. Finally the fourth model is the ACF model. This model is also an extension of the backbone+ model with the addition of the coalitions.

Regarding the feedback and diffusion theories, it is up to the modeller to introduce these two extensions. It is advised to use both of them in combination with either the backbone+, the three streams or the ACF models. This makes the dynamic more interesting and allows for the full range of effects.


%
\section{The Backbone}
\label{sec:backbone}

As mentioned before, the backbone model is the skeleton of this approach to the policy emergence. It contains several of the compatible concepts. In fact, the backbone model is the smallest policy emergence model possible. The aim with the backbone model is to have the process of policy emergence with the smallest number of actor types and interactions possible while setting the infrastructure for the other policy making theory components. For this reason, only a few compatible concepts are being considered. Only three types of actors are selected: policy makers as they are the only actors with decision making power, external parties and the electorate. Only one external party is considered. This external party is used as the link between the technical model and the policy emergence model. Within the backbone model, the interaction between the actors are limited to passive interactions. This is the case for the influence of the electorate on the policy makers. The different steps for this model are now explained and are illustrated in \autoref{fig:Conceptualisation_V3-11}.

\begin{figure}
\centering
\includegraphics[scale = 0.33, angle = 0]{figures/Conceptualisation_V3-11}
\caption{Illustration of backbone model main steps.}
\label{fig:Conceptualisation_V3-11}
\end{figure}
 
The first step is to update the policy makers with the current states of the world. This happens before the agenda setting process. The external party provides the states of the world from the technical model to the policy makers without distortion. The external party also provides the states to the electorate. The policy makers are then passively influenced by the electorate based on the electorates' beliefs as was detailed in the previous chapter.
 
Then the agenda setting process starts. Each policy maker chooses an issue that they believe is most important within the policy core beliefs. This choice is made based on the preferences associated with each of the issues in their belief tree. Using the policy makers choices, one issue is chosen for the entire system. This issue is chosen as the one being the most widely accepted amongst all policy makers. This issue is the agenda. Note that within the backbone model, there are no actor interactions.
 
During the policy formulation process, the policy makers are again the only actors present. Each policy maker chooses one policy instrument which they deem best to impact their secondary beliefs based on the issue that is on the agenda. Then the policy instrument that most of the policy makers find adequate is selected system wide. This is again obtained through the calculation of the preferences of all instruments. If this policy instrument is chosen by enough policy makers based on a predefined decision rule, it will be implemented in the technical model. This marks one cycle of the backbone model. In the next cycle and once the technical model has run with the implemented instrument, the states are once again communicated to the external party which conveys them to the policy makers and electorate. It continues then as explained.
 
All policy instruments present in the model are specified prior to the simulation by the modeller. The modeller can decide what effect each of these instruments will have on the issues in the belief tree according to the technical model used. The perception of the effect from the actors can however be widely different depending on their affiliations and the way they garner their information. The policy instruments specified can have an impact on different secondary issues at the same time. Note that, at the beginning of the model, not all the policy instruments need to be known by all actors. Some can remain hidden from the actors until they are either considered to be discovered (think of technological progress) or they are introduced to the actors through diffusion-like processes.


%
\section{The Backbone+ }
\label{sec:backbone+}

\begin{figure}
\centering
\includegraphics[scale = 0.33, angle = 0]{figures/Conceptualisation_V3-12}
\caption{Illustration of backbone+ model main steps. Note the electorate influence is not drawn but only mentionned.}
\label{fig:Conceptualisation_V3-12}
\end{figure}


%
\section{The Three Streams}
\label{sec:3S}

\begin{figure}
\centering
\includegraphics[scale = 0.33, angle = 0]{figures/Conceptualisation_V3-13}
\caption{Illustration of three streams theory model main steps. Note that several steps are not drawn but only mentionned.}
\label{fig:Conceptualisation_V3-13}
\end{figure}

\textcolor{red}{[REWRITE]}

Using the different concepts presented above and present within the backbone model, it is possible to conceptualise a model that represents the three streams theory \citep{kingdon2003agendas, zahariadis2007multiple}. In three streams theory, the agenda settiing and policy formulation processes are more complex than the mechanics outlined for the backbone. The agenda can no longer be only composed of only an issue and the policy formulation can no longer look only at an instrument. The new approach must considers the three streams: politics, policy and problem.  For this, the actors must first be allowed to choose between a policy or a problem in both steps. Once their first choice is obtained, they can look for the associated problem or policy. To allow such choice, several elements have to be modified or added to the model. First, the issues present in the belief tree are qualified as problems within the three streams approach. Second, a set of policies is added to the agenda setting step. These are devised by the modeller. These policies are conceptualised in a similar way to the belief tree. Each policy has a certain impact on specific issues at the same level of aggregation (the policy core issues) in the belief tree. Based on this impact, a preference can be set by the actor. The impact is subjective and will vary for each actor. This impact belief can also be influenced by other actors. A similar structure is established at the policy formulation step. The instruments are similar to the policies with subjective impacts. The problems are now the secondary issues in the belief tree. The instruments that can be chosen are limited to the instruments which are related to the policy on the agenda. These parent-children relations between policies and instruments are defined by the modeller and are not subjective.
 
This approach attempts to imitate the streams approach. The agents will choose whatever they find most pertinent between the policies and problems. The pertinence is defined based on calculated preferences. If they find a policy most pertinent, they will choose that and, de facto, open a policy window. They can then select an associated problem and start influencing other agents based on their selection. This happens both in the agenda setting step and the policy formulation step.
 
Throughout both steps, actors which have chosen a problem first will be able to influence other actor's beliefs on that problem. Actors that have chosen a policy first will influence other's actors belief on that policy. The actors interacting here are the so-called action actors which are the policy makers, the policy entrepreneurs and the external parties. There respective sets of actions are slightly different. This is detailed later on in this chapter. In the agenda setting process, once all these actions have been performed and similarly to the backbone, the policy and the problem that received the most interest amongst policy makers are placed on the agenda. For the policy formulation, once again each actors picks up a problem and an instrument related to the ones that have been chosen during the agenda setting stage, but at a finer resolution (lower level of abstraction). They can then interact to influence each other's belief. At the end of this round of interaction, the policy makers decide if an instrument can be implemented. This happens if the threshold to implementation has been reached amongst the policy makers.
 
Added to the individual actions and as mentioned in the literature, the actors present in the three streams theory can build teams \citep{mintrom2009policy, brouwer2011towards}. These teams are groups of actors of similar beliefs assembled to better influence other actors. These teams are short lived groups that can be a mix of policy makers, policy entrepreneurs and external parties. They are created with different actors that have a similar goal regarding a specific issue. If their attempt to convince other actors fail, then the groups will disperse. This approach represents the relatively short lived teams that come together to pass or block a specific bill or executive order for example. 


%
\section{The Advocacy Coalition Framework}
\label{sec:ACF}

\begin{figure}
\centering
\includegraphics[scale = 0.33, angle = 0]{figures/Conceptualisation_V3-14}
\caption{Illustration of ACF model main steps. Note that several steps are not drawn but only mentionned.}
\label{fig:Conceptualisation_V3-14}
\end{figure}

\textcolor{red}{[REWRITE]}

The steps used within the ACF model are closely related to the ones in the backbone. The actors considered are similar to the three streams theory: the policy makers, the policy entrepreneurs and the external parties. The states of all the actors are updated at the beginning of the model. Based on these states and on their respective aims, each of the actors can calculate their preferences for the issues in their respective belief trees. Each actor selects to advocate for the issue (s)he most prefers. For the agenda setting, they can then influence each other. At the end of this round of interaction, the policy makers select the agenda based on the issue that has been most selected. For the policy formulation, similarly to the backbone, each actor assigns a preference to an instrument based on the issue on the agenda and his/her own beliefs. After a round of interaction, the policy makers decide whether a policy instrument can be implemented or not.
 
However, there is a major addition compared to the backbone model for the ACF: coalitions. Besides individual interactions, the actors are also placed into coalitions \citep{mintrom1996advocacy}. Within the ACF, the actors do not group into teams but instead are grouped into coalitions. Coalitions are long-term entities to which actors are added based on their principle beliefs for the agenda setting and policy core beliefs for the policy formulation. Due to the nature and inertia of these higher level beliefs, these coalitions are very stable and only vary over long periods of time. Within the model, actors are added by default to coalitions with which they have most similar principle beliefs. The actions they then perform are influenced by the coalition in which they are. Furthermore, and because these coalitions are more stable, the policies and problem they advocate for coalition-wide might be different than the ones they would advocate for on a personal level. Note that the coalitions might be different in the agenda setting and the policy formulation processes.

%-
\subsection{The policy brokers}

For neutral policy brokers, the additional action that the broker has is the ability to place two actors within his network in direct contact temporarily. Conceptually, the link between the two actors joined by the policy broker gains the same level of trust as the link between the policy broker and the actors he is connecting. This increase is temporary. The aim of such action is to increase the speed of policy learning and potentially a consensus between different actors or coalitions. The policy broker needs to spend resources for such an action.


%
\section{The Feedback Theory Extension}
\label{sec:feedbackExtension}

\textcolor{red}{[REWRITE]}


%
\section{The Diffusion Theory Extension}
\label{sec:diffusionExtension}

\textcolor{red}{[REWRITE]}

The diffusion theory is an add-on theory that can be added on top of the backbone, the three streams theory or the advocacy coalition framework. It allows to connect different systems together to see whether they have an impact on each other. This is greatly affected by the different links that are present in the super-policy network. The interactions occurring in the diffusion theory happen after the intra-system interactions but before the choice of the agenda for the agenda setting and before the decision on the implementation of an instrument in the policy formulation step. These actions are provided below depending on the link between the agents. They are related to the different diffusion processes outlined in the literature \citep{berry1999innovation}.
 
\begin{enumerate}
\item Friendly: This type addresses the learning and imitation diffusion mechanisms. Within the context of this type, the actors will behave in the super-policy network similarly to the way they are behaving in the policy network. The actions are similar so as to engage in policy learning.
\item Dominant: This type addresses the normative pressure diffusion mechanism. For this type of diffusion, the actors from system 1 will actively influence the beliefs of actors of system 2. This is done on an issue per issue basis. These actions are more aggressive and different from the friendly actions.
\item Competitive: This type addresses the competitive diffusion mechanism. The competitive type will see a reverse type of action where actors from system 1 will look at actors from system 2 and change their own beliefs on the state of the world. This will have for effect that will attempt to introduce policy instruments related or stronger than the ones implemented in system 2.
\item Coercive: This type addresses the coercion diffusion mechanism. The coercive type is a stronger from of the dominant type where actors will force actors in another system into specific beliefs on specific issues.
\end{enumerate}
 
The diffusion theory does not mention any possibility of groups as seen in the three streams theory or the ACF. All actions are performed by singular agents. If there is a need for teams or coalitions, it would then be advised to change the level of aggregation of the actors and include them all into one system.


%
\section{Validation of the Three Streams Theory}
\label{sec:validation3S}
 
The approach used for the three streams theory was mostly discussed with Prof. Zahariadis considering a lot of his research is related to this theory \citep{zahariadis2007multiple, zahariadis2014ambiguity, zahariadis2003ambiguity}. Comments were targeted mostly at the concept of policy window and whether such a policy window would be observable within the three streams theory. This is also related to the selection of the problem and/or the policy by the actors present in the system. At the time of the interview, the three streams theory was using an approach where, during the agenda setting the actors would select a problem and during the policy formulation the actors would select a policy. However, and this was noted by Prof. Zahariadis, this is not how the policy window is described in the literature. The approach that was used only allowed for problem driven policy windows. It made it impossible for actors to first chose a policy and then proceed to the selection of a problem based on that policy.

Following this discussion, a considerable change was performed on the approach chosen for the three streams theory conceptualisation in the model. To better represent the policy window, it was decided that during both steps in the policy cycle, the actors would select a problem and a policy. They could decide to select first a policy or a window based on their beliefs of what is currently most important. These changes are aimed at better representing the aspect of the policy window where actors choose a problem or a policy only considering their own beliefs.
 
Additional comments related to the three streams theory approach were linked to the transparency with which the modelling assumptions are made. Prof. Zahariadis insisted on the importance of reporting on all assumptions made as they can have a major impact on the behaviour obtained from the model.
 
Comments were also made on the approach considered for the introduction of external events in the model. The conceptualisation originally argued that external event (a crisis) would lead to a larger amount of resources for the actors in the system which have to deal with this crisis. Prof. Zahariadis argued the opposite, stating that an external event would lead to a constriction of the resources available to the actors. However, this should also be accompanied by a relaxing of the threshold that define consensus between the actors. The reduction of actions allowed, along with the larger threshold for the acceptance of a policy instrument, should then lead to the faster adoption of policy instruments to deal with the external event. The approach to the external event modelling was subsequently modified to fit with this approach.
 
%
\section{Validation of the ACF}
\label{sec:validationACF}
 
Feedback on the advocacy coalition framework was mostly discussed with Prof. Weible considering a lot of his research is related to this framework \citep{weible2009themes, sabatier2014theories, nohrstedt2010logic, jenkins2014advocacy}. The feedback provided on the ACF was mostly related to the importance of documenting the concepts that are transferred into a model. This effort would be useful to understand how many of the concepts are modelled so as to represent the messiness of the policy emergence process. A comment was also made on the term deep core belief that was used throughout the conceptualisation to represent the highest belief of the actors. However, strictly related to the deep core definition that is provided in the literature, the belief level considered does not match with what a deep core belief is. The term was therefore changed to principle belief which illustrates a lower level of belief than deep core beliefs. 
